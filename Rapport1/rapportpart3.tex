\documentclass[a4paper]{article}
\usepackage[utf8]{inputenc}
\usepackage[T1]{fontenc}
\usepackage[french]{babel}
\usepackage{fancyhdr}
\usepackage {pstricks-add}
\usepackage {mathrsfs}
\usepackage{tikz}
\usepackage{amsmath}
\usepackage{amssymb}
\usepackage{amsthm}
\usepackage{ucs}
\usepackage{stmaryrd}
\usepackage{marvosym}
\usepackage{geometry}
\usepackage{setspace}
\usepackage{mathrsfs}
\usepackage{manfnt}
\usepackage{nicefrac}
\usepackage{mathpartir}
\usepackage{tikz-qtree}
\usetikzlibrary{arrows,automata}


\pagestyle{fancy}

\renewcommand{\headrulewidth}{1pt}
\fancyhead[L]{Programmation project} 
\fancyhead[R]{Marius DUFRAISSE, Manon BLANC}

\renewcommand{\footrulewidth}{1pt}
\fancyfoot[C]{\textbf{Page \thepage}} 
\fancyfoot[L]{L3 Informatique}

\title{Report: project part 3}
\author{Marius DUFRAISSE, Manon BLANC}
\date{  }


\begin{document}
	\maketitle
	\thispagestyle{fancy}
	\section{Corrections}
	
	First of all, we tried to correct the mistakes during the second part of the project. We added a little description at the beginning of each function, in order to precise what the class is supposed to do, where it is used, which classes it uses...
	
	\section{Modifications of the game}
	
	\paragraph{Handling the charts} 
	One of the required assignement was to make appear an information about the game in a chart. We decided to translate two things into charts: the money of the player and the evolution of the populations in the cities. We created two classes to deal with it : \textbf{Charts} and \textbf{SwitchCharts}. 
	\begin{itemize}
		\item \textbf{Charts }: It extends \textbf{LineChart}, its main goal is to draw the chart of the value and the derivation of it (in order to see the variations of the values), to update the charts, and to handle the bounds properly. There is also a method that can reinitialize the charts, it is used for populations' charts, because the graphs are not the same for different cities.
		\item \textbf{SwitchChart} : It enables the player to choose between a graph and the derivation. 
	\end{itemize} 
	\paragraph{Saving a game}
	\paragraph{Additional feature: Cargo destinations}
	
	\section{Known bugs}
	
	\section{Additional features}
	\paragraph{Zooming in and out}
	h
	
	
	
\end{document}